\documentclass[review]{elsarticle}

\usepackage{lineno,hyperref}
\modulolinenumbers[5]

\journal{Heliyon}

%%%%%%%%%%%%%%%%%%%%%%%
%% Elsevier bibliography styles
%%%%%%%%%%%%%%%%%%%%%%%
%% To change the style, put a % in front of the second line of the current style and
%% remove the % from the second line of the style you would like to use.
%%%%%%%%%%%%%%%%%%%%%%%

%% Numbered
%\bibliographystyle{model1-num-names}

%% Numbered without titles
%\bibliographystyle{model1a-num-names}

%% Harvard
%\bibliographystyle{model2-names.bst}\biboptions{authoryear}

%% Vancouver numbered
%\usepackage{numcompress}\bibliographystyle{model3-num-names}

%% Vancouver name/year
%\usepackage{numcompress}\bibliographystyle{model4-names}\biboptions{authoryear}

%% APA style
%\bibliographystyle{model5-names}\biboptions{authoryear}

%% AMA style
%\usepackage{numcompress}\bibliographystyle{model6-num-names}

%% `Elsevier LaTeX' style
\bibliographystyle{elsarticle-num}
%%%%%%%%%%%%%%%%%%%%%%%
%George's Additions
%%%%%%%%%%%%%%%%%%%%%%%
\usepackage{subcaption}
\usepackage{changepage}
\usepackage{amsmath}
\usepackage{mathtools}
\usepackage{amsmath,amssymb}
\newtheorem{theorem}{Theorem}[section]
\newtheorem{proposition}{Proposition}[section]
\newtheorem{corollary}{Corollary}[theorem]
\newtheorem{lemma}[theorem]{Lemma}

%% Misc. Packages
\usepackage[final]{changes}
\definechangesauthor[name={George Chacko}, color=orange]{gc}
\setremarkmarkup{(#2)}
\usepackage{listings}


\begin{document}

\begin{frontmatter}

\title{ERNIE: A Knowledge Platform to Support Research Assessment} 
%\tnotetext[mytitlenote]{This document is a collaborative effort}

%% Group authors per affiliation:
%% or include affiliations in footnotes:
\author[nl]{Samet Keserci}
\author[nl]{Avon Davey}
\author[ca]{Di Cross}
\author[gi]{Alexander R. Pico}
\author[nl]{Dmitriy Korobskiy}
\author[nl]{George Chacko \corref{cor1}}
\ead{netelabs@nete.com}

\cortext[cor1]{Corresponding author}
%\fntext[fn1]{Current address: Facebook Inc., Menlo Park, CA, USA}


\address[nl]{NETE Labs, NET ESolutions Corporation, McLean, VA, USA}
\address[ca]{Research Data Science \& Evaluation, Clarivate Analytics, USA}
\address[gi]{Gladstone Institutes, San Francisco, CA, USA}

\raggedright

\begin{abstract}

Data mining of public and commercially available data sources coupled with network analysis has been successfully used as a digital methodology to study research collaborations and knowledge flow associated with drug development. To enable quantitative studies based on this approach, we have developed Enhanced Research Network Information Environment (ERNIE), a scalable cloud-based knowledge platform that integrates free data drawn from public sources as well as licensed data from commercially available ones. Analytical workflows in ERNIE are partially automated to enable expert input at critical stages. To facilitate adoption, reuse and extensibility, ERNIE was built with open source tools. and a modular design enables the facile addition, deletion, or substitution of data sources. To demonstrate the capabilities of ERNIE, we report the results of seven case studies that span drug development, pharmacogenomics, target discovery, behavioral interventions, and opioid addiction. In these studies, we mine and analyze data from policy documents, regulatory approvals, research grants, bibliographic and patent databases, and clinical trials, to document collaborations and identify influential research accomplishments. ERNIE is a template for repositories that can be used to support expert qualitative assessments, while offering burden reduction through automation and access to integrated data.

\end{abstract}

%\begin{keyword}
%\end{keyword}

\end{frontmatter}       

\linenumbers
\raggedright

\section*{Introduction} 

The risk of research evaluation being driven by data and metrics with insufficient consideration of balanced judgment has driven the formulation of  sound principles to advise practice~\cite{LeidenManifesto2015}. However, data are still critical for evaluation, and utilitarian approaches for capture, integration, and archival that offer coverage and burden reduction assist the practice of research evaluation. While bibliographic data is usually central to research evaluation, the use of administrative records such as government statistics and research funding offers complementary value~\cite{FedStat2017}. 

Using data mining of publicly available administrative, regulatory, and scientific records and network analysis, Williams and colleagues defined relationships between scientific discoveries and major advances in medicine such as new drugs~\cite{Williams2015}. Extending this work, we have previously documented the scientific collaborations that extend across networks underlying the development of five independently developed therapeutics for cancer~\cite{Keserci2017}. Analysis of such networks supports new insights into collaboration, knowledge diffusion, and recursive learning \textit{(vide supra)}. The framework we developed to mine data and integrate it into networks for analysis incorporates expert input, is flexible, can be further automated, adapted to a range of subjects beyond drug development, and used to support individual researchers, groups, and research organizations engaged in research assessment. In addition, cloud computing and data science have evolved to the point where scalable computing services and analytical methods are accessible to a broad user population. Thus, we have developed Enhanced Research Network Informatics Environment (ERNIE), a knowledge platform that integrates publicly available data as well as data from commercial sources.  

In demonstration that is focused on the theme of data mining and network analysis, we have used the data within ERNIE to conduct seven case studies. In these studies, we mine and analyze data from policy documents, regulatory approvals, research grants, bibliographic and patent databases, and clinical trials, to document collaborations and identify influential research accomplishments. The workflows used incorporate expert knowledge into defining a set of core documents from which citations can be extracted, linked to other records in ERNIE, and analyzed. The first two case studies serve to validate ERNIE through reproducing the results of prior studies~\cite{Williams2015} on ivacaftor and ipilimumab, breakthrough drugs used to treat cystic fibrosis and melanoma. The remaining case studies concern drugs used to treat opioid addiction~\cite{Blumberg1973,Cowan1977} , a microarray system for pharmacogenomic profiling~\cite{deLeon2006}, a target discovery system originating from enzyme fragment complementation assays~\cite{Khanna1989}, and a behavioral intervention for substance abuse~\cite{Botvin1980}. The datasets that result from these studies are made available for use by other researchers and the workflows used to generate them are also archived to permit reproduction of our results, as well as review and modification of the methods used to generate them. 

\section*{Materials and Methods}

\emph{Infrastructure} ERNIE exists as two CentOS 7.4 virtual machines in the Microsoft Azure cloud with 10 terabytes of attached storage. One virtual machine, standard D8s v3 (8 vcpu, 32 Gb), houses the main database within a PostgreSQL 9.6 server, and additionally acts as a central hub for data related processes. A second virtual machine, standard DS4 (8 vcpu, 28 Gb), houses an installation of Apache Solr 7.1 that is used to create and store indexes of data relevant to ERNIE. Access to both virtual machines is made secure through use of multiple-factor authentication, with a unique time-based code verification acting as one of the key factors. SSH tunnels are used for communication between virtual machines and automated processes for ERNIE are managed through a Jenkins Continuous Integration server. Custom ETL processes mostly consisting of Bash, Python, and SQL scripts are pulled from the project Github repository~\cite{GithubERNIE2017} and deployed via Jenkins in order to load and refresh data from several sources.  

\emph{Data sources} Data in ERNIE are derived from both publicly available and commercial sources. Publicly available data are copied from the National Clinical Trials Database \cite,  the FDA Orange and Purple Books \cite,  the United States Patent Office \cite, and NIH ExPORTER \cite. Leased data are acquired from Clarivate Analytics and consist of the Web of Science Core Collection and the Derwent World Patent Index~\cite{DWPI2017}. The data in ERNIE are stored in a relational schema that is also documented in Github. \emph{[Need some text from DK and Avon!]}

\emph{Workflows} The core workflow in ERNIE consists of (i) Source document identification, in which a set of source documents relevant to the question being asked is assembled  (ii) Citation extraction, in which text references to scientific publications and patents are matched to unique identifiers such as PubMed ID, UT (Web of Science Accession Number), or US patent number.  (iii) Amplification, in which publications and patents linked by citation to extracted references are identified through citation records in the Web of Science (iv) Network analysis, in which the data assembled through the preceding stages are represented as a network and influential nodes are identified (v) Visualization of the network data to provide insight and stimulate further investigation.

\emph {Source document identification} 

\emph {Citation extraction} Input and end product

\emph {Amplification} What is the input and end product

\emph{Network analysis} Input and end product

\emph{Visualizations} Input and end product

\section*{Results and Discussion} 



\section*{Acknowledgments} Research and development reported in this publication was supported by Federal funds from the National Institute on Drug Abuse, National Institutes of Health, US Department of Health and Human Services, under Contract No. HHSN271201700053C. The content of this publication is solely the responsibility of the authors and does not necessarily represent the official views of the National Institutes of Health.


\section*{References}

\bibliography{chacko} 

\begin{figure}[!h]
%\begin{adjustwidth}{-0.5in}{0in} % Comment out/remove adjustwidth environment if table fits in text column.
\centering
\scalebox{0.99}
{
\begin{subfigure}{.5\textwidth}
  \centering
%  \includegraphics[width=.95\linewidth]{}
  \label{fig:sub1}
\end{subfigure}
\begin{subfigure}{.5\textwidth}
  \centering
 % \includegraphics[width=.95\linewidth]{}
  \label{fig:sub2}
\end{subfigure}
}
\caption{{\bf Intersecting Publications in Five Networks} Intersections were calculated across all five networks for the first generation of references (citing\_pmids) and 
as well as for the second generation of references (cited\_sids) and displayed as Venn diagrams. \emph{Left Panel.} No first generation publications are observed common to all five networks. A single publication is cited in four of five networks. \emph{Right Panel.} 14  publications are common to all five networks. Abbreviations: alem (Alemtuzumab), imat (Imatinib), nela (Nelarabine), ramu (Ramucirumab), suni(Sunitinib)}
\label{fig: test}
%\end{adjustwidth}
\end{figure}

\begin{figure}[!h]
%\begin{adjustwidth}{-0.5in}{0in} % Comment out/remove adjustwidth environment if table fits in text column.
\centering
%\scalebox{1.3}{
%\includegraphics[scale=0.1]{cy_core14_v2csv_5b.png}}
\caption{{\bf Core Publications in Networks}  The outer arcs of blue nodes identifies first generation publications (citing\_sid) for each therapeutic. Nodes in the inner ring are sized by a gradient proportion to total degree count with an upper limit of 30 and are colored by a gradient proportional to the number of drug connections (2 to 5). 14  publications are common to all five networks (Table 3) and are colored red. The remaining nodes in the inner ring connect to between 2 and 4 drugs each and are labeled accordingly. Abbreviations: alem (Alemtuzumab), imat (Imatinib), nela (Nelarabine), ramu (Ramucirumab), suni(Sunitinib).}
\label{fig2}
%\end{adjustwidth}
\end{figure}

\end{document}